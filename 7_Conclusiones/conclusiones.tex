\section{Discusión y conclusiones}\label{sec:7_conclusiones}

Para considerar que nuestro método supone un método fiable para la identificación de sistemas lente gravitatoria, el número de contrapartidas que no pueden ser explicadas por una distribución aleatoria debería ser mucho mayor que el número de asociaciones aleatorias que cumplen nuestros criterios y lo que encontramos es justo lo contrario. Los resultados de la simulación indican que siempre que tengamos una población de observaciones con una densidad como la que tenemos en la muestra, es inevitable que se formen del orden de \maths{\sim3500} emparejamientos entre observaciones que distan a \maths{\lesssim50\:\mathrm{arcsec}} que no tienen ningún tipo de interacción física. Si las medidas que caracterizan a cada observación fuesen más precisas y más exactas se haría evidente que muchos de los candidatos propuestos o bien se encuentran a una distancia demasiado grande para ser considerados candidatos a lente o bien son observaciones de un mismo objeto en los dos catálogos.

La gran mayoría de los desplazamientos al rojo pertenecientes a las observaciones de \hatlas\ que participan en el pareo (\anglicismo{matching}) han sido determinados mediante el ajuste de la SED de la galaxia \smm. Como hemos visto en la Sección~\ref{sec:3_redshift_hatlas} tenemos cierta confianza de que el ajuste es válido para estimar el \rt\ de las galaxias en formación, pero no sabemos cuantas de las observaciones de las que hemos estimado el \rt\ son en realidad observaciones de galaxias lejanas. Es posible que estemos realizando un ajuste a galaxias que en realidad se encuentran mucho más cerca. De hecho se podría explicar que encontremos más contrapartidas en la muestra real que en la simulada sin que se diera el fenómeno de lente gravitatoria; ese exceso de contrapartidas podría deberse a observaciones del mismo objeto en los dos catálogos que no hemos sido capaces de identificar como tal porque nuestra estimación del \rt\ es incorrecta.

En este trabajo hemos utilizado el método descrito en la Sección~\ref{sec:3_redshift_hatlas} para estimar el \rt\ de muchos de los objetos del catálogo \hatlas. Para considerar válida esta medida del ajuste debían cumplirse varias condiciones: no haber una estimación previa del corrimiento al rojo de esa observación, debe ser una galaxia (recordemos, según la información disponible en columna GSQ\_FLAG de este mismo catálogo) y el valor del ajuste debe encontrarse en un rango comprendido entre \maths{z=1} y \maths{z=3.5}. Opino que deberíamos buscar más indicios antes de considerar válido el ajuste realizado con ese método. 

Nuestra propuesta para identificar lentes gravitatorias se ha planteado en el supuesto de que únicamente interviene un único objeto como lente, sin embargo, en muchas ocasiones intervienen grupos de galaxias y no galaxias individuales en constitución de la lente. El criterio de que se trata de una lente gravitatoria por el hecho encontrar dos objetos con \rt\ diferentes suficientemente próximos posiblemente también sea poco realista. Partiendo de un estudio cuidadoso del entorno que se encuentran lentes gravitatorias confirmadas podríamos elaborar métodos para la identificación de entornos favorables para la presencia de lentes gravitatorias. En este sentido es posible que tengamos que identificar cúmulos de galaxias más que objetos individuales como lentes gravitatorias.

El criterio Bayesiano descrito en la Sección~\ref{sec:4_cross_identificacion} es un criterio de \cross\ de catálogos y por tanto no ha sido pensado para la identificación SLGs. Los factores bayesianos están diseñados para contrastar la hipótesis \maths{H_1} que supone los emparejados están formados por observaciones de un mismo objeto astronómico frente a la hipótesis \maths{H_2} que supone que las observaciones pertenecen a dos objetos diferentes. Deberían plantearse nuevos factores de Bayes, que permitieran contrastar por ejemplo la hipótesis de que una observación considerada pertenece a una ETGs o no, o la hipótesis de que el emparejado considerado es una lente gravitatoria frente a la hipótesis de que no lo es. Plantear estos factores de Bayes exige conocer con mayor profundidad el fenómeno de las lentes gravitatorias y la naturaleza de las ETGs, lo cual se escapa los objetivos académicos de este trabajo.

También hay lugar para mejorar nuestro método de \cross. Los factores de Bayes son quizás demasiado simples y se pueden mejorar. Las funciones densidad de \cultismo{probabilidad a priori} son funciones no informativas que se han introducido por simplicidad matemática pero cabe plantearse otras más adecuadas. Por ejemplo podríamos incorporar la información de la que disponemos sobre la distribución del \rt\ en los catálogos para mejorar el factor de Bayes fotométrico. El modelo de precisión astrométrica que se ha considerado es bastante conservador. Se ha considerado que el error astrométrico es igual a la resolución angular del telescopio, lo cuál es razonable cuando las fuentes son puntuales. Las galaxias que aparecen el los catálogos no son puntuales, tienen una extensión y en estos casos es frecuente que \maths{\sigma^p} sea inferior a la resolución angular del telescopio. Se podría haber tomado un valor de \maths{\sigma^p} menor, pero no se ha hecho por prudencia.

La resolución espacial del telescopio \h\ también juega un papel fundamental en este trabajo. Una mejora de este parámetro permitiría tener unas medidas sobre el flujo de las fuentes más precisas con una menor contaminación por parte de las fuentes próximas más débiles. Además permitiría mejorar los modelos de la SED de las galaxias tempranas y podríamos introducir mejoras en nuestro método para la estimación del \rt\ fotométrico. No solo eso. La mejora del modelo de precisión astrométrica, junto con una menor incertidumbre asociada a la posición de la fuente permitiría estudiar las estructuras formadas por las lentes gravitatorias a una escala menor, más adecuada al orden de magnitud en el que se producen estos fenómenos. Esto también reduciría la distancia máxima de emparejamiento que se ha considerado entre las fuentes y se producirían menos emparejamientos aleatorios entre observaciones de objetos que no guardan ningún tipo de relación entre sí.

Por tanto no podemos considerar nuestro método como una alternativa razonable a las ya existentes y tampoco tiene mucho sentido comparar el número de candidatos obtenidos por nuestro método con el número obtenido utilizando los métodos que encontramos en la literatura. Sin embargo se pueden plantear modificaciones en el método que quizás conduzcan a una selección de candidatos más fiable y con mayor pureza. Además podríamos considerar el método propuesto como una fase previa a un estudio más detallado porque permitiría identificar de una forma rápida, sencilla y directa, basada en un criterio estadístico riguroso, en torno al tres por ciento de los candidatos con mayor probabilidad de ser sistemas lente de entre las galaxias del catálogo \hatlas\ DR1.
\selectlanguage{spanish}

\begin{abstract}
Se ha propuesto un método para la búsqueda de galaxias en fases tempranas de su formación suponiendo que la imagen observada ha sido magnificada por el efecto de una lente gravitatoria. 
El método consiste en seleccionar un subconjunto de emparejados que surgen al aplicar un criterio estadístico de \cross\ basado en el factor de Bayes a los catálogos \hatlas\ DR1 y \gama\ DR1. El factor de Bayes utilizado consta de un término basado en la separación angular entre entre dos observaciones y otro basado en el corrimiento al rojo. En los casos en los que se cuenta con una medida fiable del corrimiento al rojo de la fuente perteneciente a \hatlas,\hspace{1mm} se utilizarán esos valores. En caso de que no se disponga, se obtendrá su \rt\ fotométrico a partir de un ajuste por mínimos cuadrados de la distribución espectral de energía de una galaxia modelo a tres medias del flujo espectral de la fuente a tres longitudes de onda concretas. Si el factor de Bayes da una alta probabilidad de que el par de observaciones no pertenecen a un mismo objeto, pero el término que depende de la distancia angular sugiere que si lo son, entonces tenemos un buen candidato para ser una lente gravitatoria.

El método desarrollado para estimar el corrimiento al rojo ha dado buenos resultados al compáralo con trabajos anteriores. Por su parte los resultados obtenidos mediante el método bayesiano para la identificación de lentes muestra una pureza relativamente baja pero resulta prometedor como fase previa a un estudio más detallado porque permite identificar de una forma rápida, sencilla y directa, basada en un criterio estadístico riguroso, entorno al tres por ciento de los candidatos con mayor probabilidad de ser sistemas lente de entre las galaxias del catálogo \hatlas\ DR1.

\end{abstract}


\textbf{Palabras clave:} Alto desplazamiento al rojo -- galaxias submilimétricas -- lente gravitatoria fuerte -- factor de Bayes -- \mbox{cross-identificación}


\selectlanguage{english}

\begin{abstract}
A method has been proposed for the search of galaxies in early phases of their formation supposing that the observed image has been magnified by the effect of a gravitational lens. The method consists of selecting a subset of paired that arise by applying a statistical criterion of cross-identification based on the Bayes factor to the catalogs \hatlas\ DR1 and \gama\ DR1. The Bayes factor used consists of a term based on the angular separation between two observations and another based on the redshift. In those cases in which a reliable measure of the redshift of the source belonging to \hatlas\ is available, these values will be used. If it is not available, its photometric redshift will be obtained from a least squares fit of the spectral energy distribution of a model galaxy to three measures of the spectral flux of the source at three specific wavelengths. If the Bayes factor gives a high probability that the pair of observations do not belong to the same object, but the term that depends on the angular distance suggests that if they are, then we have a good candidate to be a gravitational lens.

The method developed to estimate the redshift has given good results when compared to previous works. On the other hand, the results obtained by the Bayesian method for the identification of lenses show a relatively low purity but it is promising as a previous phase to a more detailed study because it allows to identify in a fast, simple and direct way, based on a rigorous statistical criterion, around the three percent of the candidates most likely to be lens systems among the galaxies in the \hatlas\ DR1 catalog.

\end{abstract}


\textbf{Key words:} high-redshift -- submillimiter galaxies -- strong gravitational lensing -- Bayes factors -- \mbox{cross-identification}
